\documentclass[12pt]{article}
\usepackage[a4paper, margin=1in]{geometry}
\usepackage{hyperref}
\usepackage{enumitem}
\usepackage{titlesec}
\usepackage{fancyhdr}

% Formatting
\setlength{\parskip}{0.8em}
\setlength{\parindent}{0pt}
\titleformat{\section}{\large\bfseries}{\thesection}{0.5em}{}

% Header setup
\pagestyle{fancy}
\fancyhf{} % clear all header/footer fields
\fancyhead[L]{Shenzhen Technology University} % left header
\fancyhead[R]{IB00398: Introduction to Reinforcement Learning} % right header
\fancyfoot[C]{\thepage} % page number in center footer

\begin{document}

\maketitle

\begin{center}
    \vspace{1em}
    \LARGE \textbf{Project Assignment 1} \\
    \vspace{1em}
    \normalsize {Instructor: Manyou Ma \quad (email: mamanyou@sztu.edu.cn)}
\end{center}

\section{Overview}
The goal of this project is for each student to identify a real-life problem and formulate it as a Markov Decision Process (MDP). This assignment emphasizes creativity in problem selection and the ability to abstract complex tasks into the RL framework. \textbf{Transition probabilities are not required at this stage.} 

\section{Learning Objectives}
By completing this assignment, you will:
\begin{itemize}[leftmargin=*]
    \item Gain practice in mapping real-life scenarios into reinforcement learning models.
    \item Identify states, actions, and reward functions in a meaningful context.
    \item Develop scientific writing skills using \LaTeX.
\end{itemize}

\section{Tasks}
\begin{enumerate}[leftmargin=*]
    \item Select a real-life problem that is complex enough to justify an MDP formulation (e.g., robotics, scheduling, traffic management, healthcare, etc.).
    \item Define the \textbf{state space}, \textbf{action space}, and \textbf{reward function}. 
    \item Write a project report in IEEE journal format, containing the following sections:
    \begin{itemize}
        \item \textbf{Introduction} 
        \begin{itemize}
            \item Motivation
            \item Related Research
        \end{itemize}
        \item \textbf{Research Problem}
        \begin{itemize}
            \item State space, actions, rewards
            \item Justification of complexity
        \end{itemize}
    \end{itemize}
    \item If you already have a faculty supervisor, consult them for problem suggestions. If not, contact the instructor, who can introduce you to local faculty or provide problem ideas.
\end{enumerate}

\section{Deliverables}
\begin{itemize}[leftmargin=*]
    \item Report in \LaTeX, formatted using the IEEE journal style.
    \item Submit both the compiled PDF and the source \texttt{.tex} file.
    \item Length: 3–4 pages (excluding references).
\end{itemize}

\section{Submission Instructions}
\begin{itemize}[leftmargin=*]
    \item \textbf{Deadline:} October 15, 2025, 23:59 (China Standard Time).
    \item \textbf{Submission Platform:} SZTU Online Learning System (link will be provided).
    \item \textbf{File Naming:} \texttt{StudentID\_Project1.pdf} and \texttt{StudentID\_Project1.tex}.
\end{itemize}

\section{Grading}
This project contributes \textbf{15\%} of the final course grade. The grading rubric is:
\begin{itemize}[leftmargin=*]
    \item Clarity of Motivation and Problem Significance: 25\%
    \item Quality of MDP Formulation: 40\%
    \item Related Research and Positioning: 15\%
    \item Report Presentation (formatting, writing, \LaTeX\ quality): 20\%
\end{itemize}

\section{Important Notes}
\begin{itemize}[leftmargin=*]
    \item Collaboration is allowed for brainstorming, but each student must submit an individual report.
    \item Plagiarism will result in a grade of zero and will be reported to the university.
    \item If you encounter difficulties with \LaTeX, sample templates will be provided.
\end{itemize}

\end{document}
